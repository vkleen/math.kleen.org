\usepackage[utf8]{inputenc}

\usepackage{amssymb}
\usepackage{amsmath}
\usepackage{amsthm}
\usepackage{thmtools}
\usepackage{etoolbox}
\usepackage{xspace}
\usepackage{mathrsfs}
\usepackage{xifthen}

\usepackage{graphicx}
\usepackage{listings}
\usepackage{fancyvrb}

\usepackage{datetime}

\usepackage{tikz}
\usetikzlibrary{arrows,decorations.markings,chains,calc,matrix}
\usepackage{tikz-cd}
\tikzset{>=cm to}

\usepackage{mathtools}

\usepackage[colorlinks=false,bookmarks]{hyperref}
\usepackage[all]{hypcap}

\usepackage[protrusion=true]{microtype}

\usepackage[style=alphabetic,citestyle=alphabetic,backref=false,hyperref=true,backend=biber]{biblatex}

\declaretheoremstyle[spaceabove=\topsep,spacebelow=\topsep,headfont=\normalfont\scshape,notefont=\normalfont\mdseries,notebraces={(}{)},bodyfont=\normalfont,postheadspace=5pt plus 1pt minus 1pt]{scdef}
\declaretheoremstyle[spaceabove=\topsep,spacebelow=\topsep,headfont=\normalfont\scshape,notefont=\normalfont\mdseries,notebraces={(}{)},bodyfont=\itshape,postheadspace=5pt plus 1pt minus 1pt]{scthm}

\declaretheorem[style=scdef,numbered=no,   name=Definition,refname={definition,definitions},Refname={Definition,Definitions}]{definition}

\declaretheorem[style=scthm,numbered=no, name=Theorem,refname={theorem,theorems},Refname={Theorem,Theorems}]{theorem}
\declaretheorem[style=scthm,numbered=no, name=Lemma,refname={lemma,lemmas},Refname={Lemma,Lemmas}]{lemma}
\declaretheorem[style=scthm,numbered=no, name=Corollary,refname={corollary,corollaries},Refname={Corollary,Corollaries}]{corollary}
\declaretheorem[style=scthm,numbered=no, name=Proposition,refname={proposition,propositions},Refname={Proposition,Propositions}]{proposition}

\undef\Re
\undef\Im

\DeclareMathOperator{\colim}{\textnormal{colim}}
\DeclareMathOperator{\Spec}{\textnormal{Spec}}
\DeclareMathOperator{\im}{\textnormal{im}}
\DeclareMathOperator{\Sing}{\textnormal{Sing}}
\DeclareMathOperator{\Map}{\textnormal{Map}}
\DeclareMathOperator{\InnHom}{\textnormal{\textbf{Hom}}}
\DeclareMathOperator{\Hom}{\textnormal{Hom}}
\undef\hom
\DeclareMathOperator{\hom}{\textnormal{hom}}
\DeclareMathOperator{\nerve}{\textnormal{N}}
\DeclareMathOperator{\simpnerve}{\textnormal{sN}}
\DeclareMathOperator{\core}{\textnormal{Core}}
\DeclareMathOperator{\id}{\textnormal{id}}
\undef\lim
\DeclareMathOperator{\lim}{\textnormal{lim}}
\DeclareMathOperator{\Op}{\textnormal{op}}
\DeclareMathOperator{\yon}{\textnormal{y}}
\DeclareMathOperator{\HomL}{\textnormal{\textbf{Hom}}^{\textnormal{L}}}
\DeclareMathOperator{\sk}{\textnormal{sk}}
\DeclareMathOperator{\cosk}{\textnormal{cosk}}
\DeclareMathOperator{\thom}{\textnormal{Th}}
\DeclareMathOperator{\Bl}{\textnormal{Bl}}
\DeclareMathOperator{\Stab}{\textnormal{Stab}}
\DeclareMathOperator{\Grp}{\textnormal{\textsf{Grp}}}
\DeclareMathOperator{\Set}{\textnormal{\textsf{Set}}}
\DeclareMathOperator{\coc}{\textnormal{\textsf{Coc}}}
\DeclareMathOperator{\class}{\textnormal{B}}
\DeclareMathOperator{\act}{\textnormal{act}}
\DeclareMathOperator{\proj}{\textnormal{proj}}
\DeclareMathOperator{\conj}{\textnormal{conj}}
\undef\div
\DeclareMathOperator{\div}{\textnormal{div}}
\DeclareMathOperator{\curl}{\textnormal{curl}}

\newcommand*{\alg}[1]{\ensuremath{\overline{#1}}\xspace}
\newcommand*{\sep}[1]{\ensuremath{#1^{\textnormal{sep}}}\xspace}
\newcommand*{\ab}[1]{\ensuremath{#1^{\textnormal{ab}}}\xspace}
\newcommand*{\units}[1]{\ensuremath{#1^\times}\xspace}
\newcommand*{\lapl}{\Delta}
\newcommand*{\grad}{\nabla}
\newcommand*{\op}[1]{\ensuremath{#1^{\textnormal{op}}}\xspace}
\newcommand*{\codiag}[1]{\ensuremath{\nabla\mkern-3mu_{#1}}\xspace}
\newcommand*{\diag}[1]{\ensuremath{\Delta_{#1}}\xspace}
\newcommand*{\objl}[1]{\textnormal{{\(#1\)}--\textsf{Obj}}}
\newcommand*{\objr}[1]{\textnormal{\textsf{Obj}--{\(#1\)}}}

\makeatletter
\newcommand*{\@old@slash}{}\let\@old@slash\slash
\def\slash{\relax\ifmmode\delimiter"502F30E\mathopen{}\else\@old@slash\fi}
\makeatother
\def\backslash{\delimiter"526E30F\mathopen{}}
\newcommand*{\tdiv}[2]{(#1\backslash #2)}

\newcommand*{\cech}{\v{C}ech\xspace}
\newcommand*{\mc}{\operatorname{\textnormal{\v{C}}}}

\newcommand*{\triv}{\textnormal{triv}}

\newcommand*{\normal}{\lhd}
\newcommand*{\isom}{\cong}
\newcommand*{\homot}{\sim}
\newcommand*{\cequiv}{\simeq}

\makeatletter
\let\@oldsubset=\subset
\def\@subsethelper#1#2{\mathrel{\raisebox{.5pt}{$$#1\@oldsubset$$}}\xspace}
\DeclareRobustCommand*{\subset}{\mathpalette\@subsethelper\relax}

\let\@oldotimes=\otimes
\def\@otimeshelper#1#2{\mathrel{\raisebox{.5pt}{$$#1\@oldotimes$$}}\xspace}
\DeclareRobustCommand*{\otimes}{\mathpalette\@otimeshelper\relax}
\makeatother

\tikzset{/tikz/commutative diagrams/arrows={thin}}

\makeatletter
\newbox\@xrat
\renewcommand*{\xrightarrow}[2][-cm to]{%
  \setbox\@xrat=\hbox{\ensuremath{\scriptstyle #2}}
  \pgfmathsetlengthmacro{\@xratlen}{max(1.6em,\wd\@xrat+.6em)}
  \pgfmathsetlengthmacro{\@xratinnersep}{.5ex-\dp\@xrat}
  \mathrel{\tikz [#1,baseline=-.6ex]
    \draw (0,0) -- node[auto,inner sep=\@xratinnersep] {\box\@xrat} (\@xratlen,0) ;}}
\renewcommand*{\xleftarrow}[2][cm to-]{%
  \setbox\@xrat=\hbox{\ensuremath{\scriptstyle #2}}
  \pgfmathsetlengthmacro{\@xratlen}{max(1.6em,\wd\@xrat+.6em)}
  \pgfmathsetlengthmacro{\@xratinnersep}{.5ex-\dp\@xrat}
  \mathrel{\tikz [#1,baseline=-.6ex]
    \draw (0,0) -- node[auto,inner sep=\@xratinnersep] {\box\@xrat} (\@xratlen,0) ;}}
\newcommand*{\xrightarrowb}[2][-cm to]{%
  \setbox\@xrat=\hbox{\ensuremath{\scriptstyle #2}}
  \pgfmathsetlengthmacro{\@xratlen}{max(1.6em,\wd\@xrat+.6em)}
  \pgfmathsetlengthmacro{\@xratinnersep}{.5ex}
  \mathrel{\tikz [#1,baseline=-.6ex]
    \draw (0,0) -- node[auto,inner sep=\@xratinnersep] {\box\@xrat} (\@xratlen,0) ;}}
\newcommand*{\xleftarrowb}[2][cm to-]{%
  \setbox\@xrat=\hbox{\ensuremath{\scriptstyle #2}}
  \pgfmathsetlengthmacro{\@xratlen}{max(1.6em,\wd\@xrat+.6em)}
  \pgfmathsetlengthmacro{\@xratinnersep}{.5ex}
  \mathrel{\tikz [#1,baseline=-.6ex]
    \draw (0,0) -- node[auto,inner sep=\@xratinnersep] {\box\@xrat} (\@xratlen,0) ;}}

\pgfarrowsdeclare{my right hook}{my right hook}
{
\arrowsize=0.2pt
\advance\arrowsize by .5\pgflinewidth
\pgfarrowsleftextend{-.5\pgflinewidth}
\pgfarrowsrightextend{3.5\arrowsize+.5\pgflinewidth}
}
{
\arrowsize=0.2pt
\advance\arrowsize by .5\pgflinewidth
\pgfsetdash{}{0pt} % do not dash
\pgfsetroundjoin % fix join
\pgfsetroundcap % fix cap
\pgfpathmoveto{\pgfpoint{0\arrowsize}{-7\arrowsize}}
\pgfpatharc{-90}{90}{3.5\arrowsize}
\pgfusepathqstroke
}

\tikzset{%
  iso/.style={above,sloped,inner sep=0},
  iso'/.style={below,sloped,inner sep=0},
  to/.style={-cm to},
  onto/.style={-cm double to},
  into/.style={my right hook-cm to},
  mapsto/.style={|-cm to},
  clim/.style={decoration={markings,
                           mark=at position#1 with {\draw[-] (0,-3\pgflinewidth) -- (0,3\pgflinewidth);}},
               postaction=decorate},
  clim/.default=0.5,
  opim/.style={decoration={markings,
                           mark=at position#1 with {\draw[-] circle(3\pgflinewidth);}},
               postaction=decorate},
  opim/.default=0.5
}

\newcommand*\@tikzto[3]{\mathrel{\begin{tikzpicture}[baseline]%
      \draw[to,line width={#2\pgflinewidth},scale=#1](0,.55ex) -- (#3,.55ex);%
    \end{tikzpicture}}}

\newcommand*\@tikzonto[3]{\mathrel{\begin{tikzpicture}[baseline]%
      \draw[onto,line width={#2\pgflinewidth},scale=#1](0,.55ex) -- (#3,.55ex);%
    \end{tikzpicture}}}

\newcommand*\@tikzinto[3]{\mathrel{\begin{tikzpicture}[baseline]%
      \draw[into,line width={#2\pgflinewidth},scale=#1](0,.55ex) -- (#3,.55ex);%
    \end{tikzpicture}}}

\newcommand*\@tikzclim[3]{\mathrel{\begin{tikzpicture}[baseline]%
      \draw[into,clim,line width={#2\pgflinewidth},scale=#1](0,.55ex) -- (#3,.55ex);%
    \end{tikzpicture}}}

\newcommand*\@tikzopim[3]{\mathrel{\begin{tikzpicture}[baseline]%
      \draw[into,opim,line width={#2\pgflinewidth},scale=#1](0,.55ex) -- (#3,.55ex);%
    \end{tikzpicture}}}

\newcommand*\@tikzmapsto[3]{\mathrel{\begin{tikzpicture}[baseline]%
      \draw[mapsto,line width={#2\pgflinewidth},scale=#1](0,.55ex) -- (#3,.55ex);%
    \end{tikzpicture}}}

\newcommand*\@tikziso[4]{\mathrel{\begin{tikzpicture}[baseline]%
      \draw[to,line width={#2\pgflinewidth},scale=#1](0,.55ex) -- node[iso,pos=0.47,inner sep=#4]{$$#3\sim$$} (1.6em,.55ex);%
    \end{tikzpicture}}}

\newcommand*\tikzto{\mathchoice%
  {\@tikzto{1.0}{1}{1.6em}}%
  {\@tikzto{1.0}{1}{1.6em}}%
  {\@tikzto{0.8}{0.9}{0.8em}}%
  {\@tikzto{0.6}{0.75}{0.8em}}}
\newcommand*\tikzonto{\mathchoice%
  {\@tikzonto{1.0}{1}{1.6em}}%
  {\@tikzonto{1.0}{1}{1.6em}}%
  {\@tikzonto{0.8}{0.9}{0.8em}}%
  {\@tikzonto{0.6}{0.75}{0.8em}}}
\newcommand*\tikzinto{\mathchoice%
  {\@tikzinto{1.0}{1}{1.6em}}%
  {\@tikzinto{1.0}{1}{1.6em}}%
  {\@tikzinto{0.8}{0.9}{0.8em}}%
  {\@tikzinto{0.6}{0.75}{0.8em}}}
\newcommand*\tikzclim{\mathchoice%
  {\@tikzclim{1.0}{1}{1.6em}}%
  {\@tikzclim{1.0}{1}{1.6em}}%
  {\@tikzclim{0.8}{0.9}{0.8em}}%
  {\@tikzclim{0.6}{0.75}{0.8em}}}
\newcommand*\tikzopim{\mathchoice%
  {\@tikzopim{1.0}{1}{1.6em}}%
  {\@tikzopim{1.0}{1}{1.6em}}%
  {\@tikzopim{0.8}{0.9}{0.8em}}%
  {\@tikzopim{0.6}{0.75}{0.8em}}}
\newcommand*\tikzmapsto{\mathchoice%
  {\@tikzmapsto{1.0}{1}{1.6em}}%
  {\@tikzmapsto{1.0}{1}{1.6em}}%
  {\@tikzmapsto{0.8}{0.9}{0.8em}}%
  {\@tikzmapsto{0.6}{0.75}{0.8em}}}
\newcommand*\tikziso{\mathchoice{\@tikziso{1.0}{1}{\displaystyle}{0pt}}%
  {\@tikziso{1.0}{1}{\textstyle}{0pt}}%
  {\@tikziso{0.8}{0.9}{\scriptstyle}{0pt}}%
  {\@tikziso{0.67}{0.8}{\scriptscriptstyle}{0.15ex}}}
\makeatother

\renewcommand*{\to}[1][]{\ifthenelse{\isempty{#1}}{\tikzto}{\xrightarrowb{#1}}}
\newcommand*{\into}[1][]{\ifthenelse{\isempty{#1}}{\tikzinto}{\xrightarrowb[into]{#1}}}
\newcommand*{\onto}[1][]{\ifthenelse{\isempty{#1}}{\tikzonto}{\xrightarrowb[onto]{#1}}}
\newcommand*{\clim}{\tikzclim}
\newcommand*{\opim}{\tikzopim}

\newcommand*{\iso}{\tikziso}

\renewcommand*{\mapsto}{\tikzmapsto}

\newcommand*{\ca}[1]{\ensuremath{\mathscr{#1}}\xspace}
\renewcommand*{\cal}[1]{\ensuremath{\mathcal{#1}}\xspace}
\newcommand*{\f}[1]{\ensuremath{\mathfrak{#1}}\xspace}

\newcommand{\cl}[2][0]{{}\mkern#1mu\overline{\mkern-#1mu#2}}
\newcommand*{\Int}[1]{\ensuremath{#1^\circ}\xspace}

\newcommand*{\infcat}[1][1]{\((\infty,#1)\)–category\xspace}
\newcommand*{\linfcat}{\(\infty\)–category\xspace}
\newcommand*{\infcats}[1][1]{\((\infty,#1)\)–categories\xspace}
\newcommand*{\linfcats}{\(\infty\)–categories\xspace}
\newcommand*{\Cat}{\textnormal{\textsf{Cat}}}
\newcommand*{\spc}{\textnormal{\textbf{Spc}}}
\newcommand*{\ICat}{\textnormal{\textbf{Cat}}}
\newcommand*{\infpshv}{\operatorname{\textnormal{\textbf{PShv}}}}
\newcommand*{\infshv}{\operatorname{\textnormal{\textbf{Sh}}}}
\newcommand*{\hypshv}{\operatorname{\widehat{\textnormal{\textbf{Sh}}}}}
\newcommand*{\hyp}[1][]{\operatorname{\ca H}^{#1}\mkern-2mu}
\newcommand*{\phyp}[1][]{\operatorname{\ca H}^{#1}_{\bullet}\mkern-2mu}
\newcommand*{\sh}{\operatorname{\ca{SH}}}
\newcommand*{\shv}{\operatorname{\textnormal{\textsf{Shv}}}}
\newcommand*{\pshv}{\operatorname{\textnormal{\textsf{PShv}}}}
\newcommand*{\Aff}{\operatorname{\textnormal{\textsf{Aff}}}}
\newcommand*{\Ring}{\operatorname{\textnormal{\textsf{Ring}}}}
\newcommand*{\Alg}{\operatorname{\textnormal{\textsf{Alg}}}}
\newcommand*{\Disc}{\operatorname{\textnormal{\textsf{Disc}}}}
\newcommand*{\mfd}{\operatorname{\textnormal{\textsf{Mfd}}}}
\newcommand*{\sm}[1][k]{\mathop{\textnormal{\textsf{Sm}}/#1}}
\newcommand*{\Simp}{\mathbf{\Delta}}
\newcommand*{\sset}{\operatorname{\textnormal{\textsf{sSet}}}}
\newcommand*{\ssetQ}{\operatorname{\textnormal{\textsf{sSet}}_{\textnormal{\textsf{Q}}}}}
\newcommand*{\ssetJ}{\operatorname{\textnormal{\textsf{sSet}}_{\textnormal{\textsf{J}}}}}
\newcommand*{\Top}{\operatorname{\textnormal{\textsf{Top}}}}
\newcommand*{\qcat}{\operatorname{\textnormal{\textsf{qCat}}_{2}}}
\newcommand*{\Fin}{\operatorname{\textnormal{\textsf{Fin}}_{*}}}
\newcommand*{\tang}[1]{\textnormal{T}\mkern-2mu_{#1}}
\newcommand*{\dual}[1]{\textnormal{D}#1}
\newcommand*{\sch}[1][k]{\mathop{\textnormal{\textsf{Sch}}/#1}}
\newcommand*{\tors}[1][G]{\mathop{\textnormal{Tors}_{#1}}}
\newcommand*{\Tors}[1][G]{\mathop{\textnormal{\textsf{Tors}}_{#1}}}

\def\<#1>{\left\langle #1 \right\rangle}

\newcommand{\dd}[1][]{%
  \mathrm{d}%
  \if\relax\detokenize{#1}\relax%
  \else%
    ^{#1}\mkern-1.5mu%
  \fi%
}

\newcommand{\Diff}[1][]{%
  \mathrm{D}%
  \if\relax\detokenize{#1}\relax%
  \else%
  ^{#1}\m
  kern-1.5mu%
  \fi%
}

\newcommand{\ddd}[1][]{%
  \mathop{}\!\dd[#1]%
}

\undef\AA
\undef\SS
\renewcommand*{\do}[1]{\expandafter\def\csname#1#1\endcsname{\ensuremath{\mathbb{#1}}\xspace}}
\docsvlist{A,B,C,D,E,F,G,H,I,J,K,L,M,N,O,P,Q,R,S,T,U,V,W,X,Y,Z}

\shortdate
\def\parsedate#1/#2/#3\endparse{%
  \date{\formatdate{#3}{#2}{#1}}
}
\newcommand{\setdate}[1]{%
  \ifthenelse{\isempty{#1}}{}{\parsedate#1\endparse}
}
